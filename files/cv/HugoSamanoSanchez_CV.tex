%%%%%%%%%%%%%%%%%%%%%%%%%%%%%%%%%%%%%%%%%
% "ModernCV" CV and Cover Letter
% LaTeX Template
% Version 1.11 (19/6/14)
%
% This template has been downloaded from:
% http://www.LaTeXTemplates.com
%
% Original author:
% Xavier Danaux (xdanaux@gmail.com)
%
% License:
% CC BY-NC-SA 3.0 (http://creativecommons.org/licenses/by-nc-sa/3.0/)
%
% Important note:
% This template requires the moderncv.cls and .sty files to be in the same 
% directory as this .tex file. These files provide the resume style and themes 
% used for structuring the document.
%
%%%%%%%%%%%%%%%%%%%%%%%%%%%%%%%%%%%%%%%%%

%----------------------------------------------------------------------------------------
%	PACKAGES AND OTHER DOCUMENT CONFIGURATIONS
%----------------------------------------------------------------------------------------

\documentclass[11pt,a4paper,sans]{moderncv} % Font sizes: 10, 11, or 12; paper sizes: a4paper, letterpaper, a5paper, legalpaper, executivepaper or landscape; font families: sans or roman

\moderncvstyle{classic} % CV theme - options include: 'casual' (default), 'classic', 'oldstyle' and 'banking'
\moderncvcolor{black} % CV color - options include: 'blue' (default), 'orange', 'green', 'red', 'purple', 'grey' and 'black'

\usepackage{lipsum} % Used for inserting dummy 'Lorem ipsum' text into the template
\usepackage{amsmath}
\usepackage[english]{babel}
\usepackage{enumitem}


\usepackage[scale=.85]{geometry} % Reduce document margins
\setlength{\hintscolumnwidth}{2.5cm} % Uncomment to change the width of the dates column
%\setitemize{label=\textbullet}
\setitemize{label=\raisebox{.5\height}{\scalebox{0.8}{\textbullet}}}
\setlength{\itemsep}{0.1pt}
\setlength{\parskip}{0pt}
\setlength{\parsep}{0pt}
%\setlength{\makecvtitlenamewidth}{10cm} % For the 'classic' style, uncomment to adjust the width of the space allocated to your name

%----------------------------------------------------------------------------------------
%	NAME AND CONTACT INFORMATION SECTION
%----------------------------------------------------------------------------------------

\firstname{Hugo C.} % Your first name
\familyname{S\'amano-S\'anchez} % Your last name

% All information in this block is optional, comment out any lines you don't need
%\title{\emph{Curriculum Vitae (\today)}} % with date
\title{\emph{Curriculum Vitae}}
%\address{H}{Heidelberg, Germany, 69126}
%\mobile{+86 - 11111111111}
\email{samano@intl.zju.edu.cn}
%\email{hsamano@lcg.unam.mx}
%\mobile{+86 - 1111111111}
%\fax{+86 (0) 1111111111}
%
\extrainfo{
  \href{https://hugocarlos.github.io}{hugocarlos.github.io}\\
  \href{https://scholar.google.de/citations?user=c5EwcVAAAAAJ}{Google Scholar: c5EwcVAAAAAJ}\\
  \href{http://orcid.org/0000-0003-4744-4787}{ORCID: 0000-0003-4744-4787}\\
%  \href{https://publons.com/a/1644581/}{Publons: 1644581}\\
  \href{https://twitter.com/hugocarlos__}{Twitter: @hugocarlos\textunderscore\textunderscore}
}

%\email{\date{}}
%\extrainfo{Birth date: November 8th, 19xx.}
%\photo[70pt][0pt]{Hugo_Samano.jpg} % The first bracket is the picture height, the second is the thickness of the frame around the picture (0pt for no frame)


%----------------------------------------------------------------------------------------
\begin{document}
\makecvtitle % Print the CV title
\vspace{-1.2cm}
%----------------------------------------------------------------------------------------
%	EDUCATION SECTION
%----------------------------------------------------------------------------------------
\section{Summary}
Mostly interested in computational approaches to understand host-pathogen interactions. Trained in Genomic Sciences I soon got into bioinformatics and computational biology. After the university I worked on antibody repertoires to compare immunologic challenges. In 2013 I started a masters at the University of Basel and performed my thesis at the National University of Singapore working with lipidomics data from clinical isolates of malaria patients. I defended my PhD thesis at the University of Heidelberg, after working with Short Linear Motifs at the host-pathogen interface at EMBL Heidelberg. Currently, I work as a lecturer at the Biomedical Informatics bachelor's programme from the Zhejiang University-University of Edinburgh Institute in China.

\vspace{0.5cm}

\section{Education}

\cventry{10/15 - 12/19}{Ph.D. in Molecular Biology}{\emph{European Molecular Biology Laboratory / University of Heidelberg}}{Heidelberg, Germany}{}{Magna cum laude.}
\cventry{09/13 - 02/15}{Joint Master Programme in Infectious Diseases, Vaccinology and Drug Discovery}{\emph{University of Basel, Switzerland and the National University of Singapore, Singapore}}{}{}{}
\cventry{08/07 - 06/11}{B.Sc. in Genomics Sciences}{\emph{National Autonomous University of Mexico}}{Cuernavaca, Mexico}{}{With honours.}

\vspace{0.5cm}

\section{Specific Training}
\cventry{November 2019}{ELIXIR Instructor Training workshop}{\emph{European Molecular Biology Laboratory}}{Heidelberg, Germany}{}{}
\cventry{January 2019}{EICAT Complementary Scientific Skills Training: Scientific Presentations}{\emph{European Molecular Biology Laboratory}}{Heidelberg, Germany}{}{}

\vspace{0.5cm}

%----------------------------------------------------------------------------------------
%	WORK EXPERIENCE SECTION
%----------------------------------------------------------------------------------------

\section{Professional Experience}
\cventry{10/21 - current}{Faculty}{}{Zhejiang University-University of Edinburgh Institute}{Haining, China}{}
\cventry{05/20 - 09/21}{Lecturer}{Biomedical Informatics bachelor's programme}{Zhejiang University-University of Edinburgh Institute}{Haining, China}{}
\cventry{10/15 - 01/20}{PhD Student / Bridging Postdoctoral Fellow}{}{European Molecular Biology Laboratory}{Heidelgerg, Germany}{
  \begin{itemize}
    \item Advisor: Dr. Toby J. Gibson.
    \item Computational screening for Eukaryotic Short Linear Motifs mimicked by bacterial proteins. Five publications.
    \item Computational prediction of Linear Motifs at the \textit{Plasmodium}-Host interface.
  \end{itemize}
}
\cventry{01/14 - 02/15}{Master thesis}{}{Vivax Malaria Lab, National University of Singapore}{Singapore}{
  \begin{itemize}
  \item Advisors: Dr. Bruce M. Russell and Prof. Dr. Markus R. Wenk.
  \item Data analysis of Mass Spectrometry and Tandem Mass Spectrometry experiments to identify characteristic lipids of \textit{Plasmodium vivax}-infected human reticulocytes. Thesis.
  \item Bioinformatics analyses to study receptor-ligand interactions during \textit{Plasmodium vivax} reticulocyte infection. One publication.
  \end{itemize}
}
\cventry{10/12 - 07/13}{Research Assistant}{}{National Institute of Public Health}{Cuernavaca, Mexico}{
  \begin{itemize}
  \item Advisor: Dr. Jesus Martinez-Barnetche.
  \item Bioinformatics analysis of high-throughput sequencing data from human and murine lymphocyte antigen receptors challenged with inactivated viruses or bacterial pathogens, respectively. Three publications.
  \end{itemize}
}
\cventry{01/12 - 09/12}{Bioinformatics Consultant}{}{Winter Genomics (\href{http://www.wintergenomics.com}{http://www.wintergenomics.com})}{Cuernavaca, Mexico}{
  \begin{itemize}
  \item Advisor: Dr. Enrique Morett.
  \item Genome assembly, annotation and structural variation analysis of two \textit{Babesia} species under two growth conditions. Bioinformatics analysis service at the Winter Genomics start-up.
  \end{itemize}
}
\cventry{08/10 -- 12/11}{Undergraduate Research Assistant}{}{Center for Genomic Sciences, National Autonomous University of Mexico}{Cuernavaca, Mexico}{
  \begin{itemize}
  \item Advisor: Dr. Esperanza Martinez-Romero.
  \item Phylogenetic analysis of cospeciation events between scale insects and their symbiotic bacteria. One publication.
  \end{itemize}
}

\vspace{0.5cm}

\section{Appointments}
\cventry{05/22 - current}{PhD Supervisor}{College of Medicine and Veterinary Medicine}{University of Edinburgh}{Scotland}{}
\cventry{10/21 - current}{Lecturer}{Biomedical Sciences and Biomedical Informatics bachelor programmes}{Zhejiang University-University of Edinburgh Institute}{Haining, China}{}
\cventry{09/21 - current}{Honorary Lecturer}{College of Medicine and Veterinary Medicine}{University of Edinburgh}{Scotland}{}

\vspace{0.5cm}

\section{Scientific Publications}
\vspace{-.1cm}
\footnotesize{$^{\ast}$ Contributed equally.} \hspace{.5cm}
%\footnotesize{$^{\dagger}$ Shared last authorship.}
\vspace{.1cm}
\begin{itemize}
  \item \textbf{\underline{S\'amano-S\'anchez, H.}}, Gibson, T.J., Chemes, L.B., Using Linear Motif Database Resources to Identify SH2 Domain Binders. \textit{Methods Mol Biol}, 2023.
  \href{https://doi.org/10.1007/978-1-0716-3393-9_9}{doi: 10.1007/978-1-0716-3393-9\_9}
  \item Kumar, K., Michael, S., Alvarado-Valverde, J., Meszaros, B., \textbf{\underline{S\'amano-S\'anchez, H.}}, Zeke, A., Dobson, L., Lazar, T., Ord, M., Nagpal, A., Farahi, N., Kaser, M., Kraleti, R., Davey, N.E., Pancsa, R., Chemes, L.B., Gibson, T.J., The eukaryotic linear motif resource: 2022 release. \textit{Nucleic Acids Research}, 2022. \href{https://doi.org/10.1093/nar/gkab975}{doi: 10.1093/nar/gkab975}
  \item Mezsaros, B., \textbf{\underline{S\'amano-S\'anchez, H.}}, Alvarado-Valverde, J., Calyseva, J., Martinez-Perez, E., Alves, R., Shields, D.C., Kumar, M., Rippmann, F., Chemes, L.B., Gibson, T.J. Short linear motif candidates in the cell entry system used by SARS-CoV-2 and their potential therapeutic implications. \textit{Science Signaling}, 2021. \href{https://doi.org/10.1126/scisignal.abd0334}{doi: 10.1126/scisignal.abd0334}
  \item \textbf{\underline{S\'amano-S\'anchez, H.}}, Gibson, T.J. Mimicry of Short Linear Motifs by Bacterial Pathogens: A Drugging Opportunity. \textit{Trends in Biochemical Sciences}, 2020. \href{https://doi.org/10.1016/j.tibs.2020.03.003}{doi: 10.1016/j.tibs.2020.03.003}
  \item Kumar, M., Gouw, M., \textbf{\underline{S\'amano-S\'anchez, H.}}, Pancsa, R., Glavina, J., Diakogianni, A., Alvarado-Valverde, J., Bukirova, D., Calyseva, J., Palopoli, N., Davey, N.E., Chemes, L.B., Gibson, T.J. ELM - the eukaryotic linear motif resource in 2020. \textit{Nucleic Acids Research}, 2019. \href{https://doi.org/10.1093/nar/gkz1030}{doi: 10.1093/nar/gkz1030}
  \item Sampietro, D., \textbf{\underline{S\'amano-S\'anchez, H.}}, Davey, N.E., Sharan, M., Meszaros, B., Gibson, T.J., Kumar, M. Conserved SQ and QS motifs in bacterial effectors suggest pathogen interplay with the ATM kinase family during infection. \textit{biorXiv}, 2018. \href{https://doi.org/10.1101/364117}{doi: 10.1101/364117}
  \item Gouw, M., Michael, S., \textbf{\underline{S\'amano-S\'anchez, H.}}, Kumar, M., Zeke, A., Lang, B., Bely B., Chemes, L.B., Davey, N.E., Deng, Z., Diella, F., Gurth, C.-M., Huber, A.-K., Kleinsorg, S., Schlegel, L.S., Palopoli, N., Roey, K.V., Altenberg, B., Remenyi, A., Dinkel, H., Gibson, J.T. The eukaryotic linear motif resource - 2018 update. \textit{Nucleic Acids Research}. \href{https://doi.org/10.1093/nar/gkx1077}{doi: 10.1093/nar/gkx1077}
  \item Kosaisavee, V., Suwanarusk, R., Chua, A.C.Y., Kyle, D.E., Malleret, B., Zhang, R., Imwong, M., Imerbsin, R., Ubalee, R., \textbf{\underline{S\'amano-S\'anchez, H.}}, Yeung, B.K.S., Ong, J.J.Y., Lombardini, E., Nosten, F., Tan, K.S.W., Bifani, P., Snounou, G., R\'enia, L., Russell, B. Strict tropism for CD71+/CD234+ human reticulocytes limits the zoonotic potential of \textit{Plasmodium cynomolgi}. \textit{Blood}, 2017. \href{https://doi.org/10.1182/blood-2017-02-764787}{doi: 10.1182/blood-2017-02-764787}
  \item Gouw, M., \textbf{\underline{S\'amano-S\'anchez, H.}}, Roey, K.V., Diella, F., Gibson, T.J., Dinkel, H. Exploring Short Linear Motifs Using the ELM Database and Tools. \textit{Current Protocols in Bioinformatics}, 2017. \href{https://doi.org/10.1002/cpbi.26}{doi: 10.1002/cpbi.26}
  \item Godoy-Lozano, E.E., Tellez-Sosa, J., Sanchez-Gonzalez, G., \textbf{\underline{S\'amano-S\'anchez, H.}}, Aguilar-Salgado, A., Salinas-Rodriguez, A., Cortina-Ceballos, B., Vivanco-Cid, H., Hernandez-Flores, K., Pfaff, J.M., Kahle, K.M., Doranz, B.J., Gomez-Barreto, R.E., Valdovinos-Torres, H., Lopez-Martinez, I., Rodriguez, M.H., Mart\'inez-Barnetche, J. Lower IgG somatic hypermutation rates during acute dengue virus infection is compatible with a germinal center-independent B cell response. \textit{Genome Medicine}, 2016. \href{https://doi.org/10.1186/s13073-016-0276-1}{doi: 10.1186/s13073-016-0276-1}
  \item Cortina-Ceballos, B., Godoy-Lozano, E.E., Tellez-Sosa, M., Ovilla-Munoz, M., \textbf{\underline{S\'amano-S\'anchez, H.}}, Aguilar-Salgado, A., Gomez-Barreto, R.E., Valdovinos-Torres, H., Lopez-Martinez, I., Aparicio-Antonio, R., Rodriguez, M.H., Mart\'inez-Barnetche, J. Longitudinal analysis of the peripheral B cell repertoire reveals unique effects of immunization with a new Influenza virus strain. \textit{Genome Medicine}, 2015. \href{https://doi.org/10.1186/s13073-015-0239-y}{doi: 10.1186/s13073-015-0239-y}
  \item Cortina-Ceballos, B.\textbf{*}, Godoy-Lozano, E.E.\textbf{*}, \textbf{\underline{S\'amano-S\'anchez, H.}}\textbf{*}, Aguilar-Salgado, A., Velasco-Herrera, M.D.C., Vargas-Chavez, C., Velazquez-Ramirez, D., Romero, G., Moreno, J., Tellez-Sosa, J., Mart\'inez-Barnetche, J. Reconstructing and mining the B cell repertoire with ImmunediveRsity. \textit{mAbs}, 2015. \href{https://doi.org/10.1080/19420862.2015.1026502}{doi: 10.1080/19420862.2015.1026502}
  \item Rosenblueth, M., Sayavedra, L., \textbf{\underline{S\'amano-S\'anchez, H.}}, Roth, A., Mart\'inez-Romero, E. Evolutionary Relationships of Flavobacteria and Enterobacterial Endosymbionts with their Scale Insect Hosts (Hemiptera: Coccoidea). \textit{Journal of Evolutionary Biology}, 2011. \href{https://doi.org/10.1111/j.1420-9101.2012.02611.x}{doi: 10.1111/j.1420-9101.2012.02611.x}\\
\end{itemize}
%\pagebreak

\vspace{0.5cm}

\section{Awards and Scholarships}
\begin{itemize}
  \item Second prize at the Teaching Competition for young faculty of Zhejiang University-University of Edinburgh Institute (2022).
  \item Travel Fellow Award from the \textit{Funda\c c\~ao de Amparo \`a Pesquisa do Estado de S\~ao Paulo} to participate in the Science of Eradication: Malaria course in Sao Paulo, Brazil (2015).
  \item Travel Fellow Award from the National Institute of Allergy and Infectious Diseases to attend the 5th International Conference of Research on \textit{Plasmodium vivax} Malaria (2015).
  \item Scholarship from the Swiss Tropical and Public Health Institute to study the Joint master program on Infectious Diseases, Vaccinology and Drug Discovery (2013-2015).\\
\end{itemize}
%\vspace{-.13cm}
%\section{Software development}
%\vspace{-.1cm}
%\begin{itemize}
%\item \href{http://www.bioconductor.org/packages/release/bioc/html/xxx.html}{XxxXxx}: Xxx. \textit{R/Bioconductor}.
%\end{itemize}
%\vspace{-.2cm}

\vspace{0.5cm}

\section{Talks and Posters}
\vspace{-.1cm}
\textbf{Invited talks}
\begin{itemize}
  \item Talk to the Parasitology Unit of the Center for Infectious Diseases, Heidelberg University Hospital, Heidelberg, Germany, 2019.
  \item Institutional seminar at the Biotechnology Research Institute (IIB), National University of San Martin (UNSAM), Buenos Aires, Argentina, 2017.
\end{itemize}
\vspace{.1cm}
\textbf{Selected talks}
\begin{itemize}
  \item Cold Spring Harbor Asia Conference: Bacterial Infection and Host Defense, Suzhou, China, 2019.
  \item VIII Argentinian Bioinformatics and Computational Biology Congress, Posadas, Argentina, 2017.
  \item 5th International Conference of Research on \textit{Plasmodium vivax} Malaria, Bali, Indonesia, 2015.
\end{itemize}
\vspace{.1cm}
\textbf{Poster presentations}
\begin{itemize}
  \item BioMalPar XV: Biology and Pathology of the Malaria Parasite, Heidelberg, Germany, 2019.
  \item EMBO Workshop on Molecular advances and parasite strategies in host infection, Les Embiez Island, France, 2018.
  \item EMBO Conference on Hijacking host signalling and epigenetic mimicry during infections, Paris, France, 2017.
  \item 2nd Interdisciplinary signaling workshop, Visegrad, Hungary, 2017.
  \item The modularity of signaling proteins and networks, Seefeld, Austria, 2016.\\
\end{itemize}

%------------------------------------------------
%\vspace{0.5cm}
\pagebreak

\section{Teaching}
\vspace{-.1cm}
\textbf{Mentor}
\begin{itemize}
  \item PhD assistant supervisor of Jiayuan Chen, University of Edinburgh Biomedical Sciences programme, 2022-current
  \item PhD co-supervisor of Xiaomeng Li, dual degree ZJU-UoE Integrative Biomedical Sciences programme, 2021-current
  \item PhD co-supervisor of Shengjie Jin, University of Edinburgh Biomedical Sciences programme, 2021-current
  \item PhD assistant supervisor of Chenyu Wang, University of Edinburgh Biomedical Sciences programme, 2021-current
  \item Master thesis supervisor of Davide Sampietro, Erasmus student from Universit\'a deli Studi di Milano - Bicocca, 2018.
\end{itemize}
\vspace{.1cm}
\textbf{Course Organizer}
\begin{itemize}
  \item Biomedical Informatics 3 (BMI3), Biomedical Informatics Programme, Zhejiang University-University of Edinburgh Institute, China, 2023. This is a core course on algorithm design.
  \item Integrative Biomedical Sciences 3 (IBMS3), Biomedical Informatics and Biomedical Sciences Programmes, Zhejiang University-University of Edinburgh Institute, China, 2022. This is a core course on career development and experimental design.
  \item Bioinformatics Tools for Protein Structure, Disorder and Interaction Analysis, National University of San Martin, Buenos Aires, Argentina, 2017.
\end{itemize}
\vspace{.1cm}
\textbf{Lecturer}
\begin{itemize}
  \item Advanced Mathematics 1 (AM1), Introduction to Biomedical Informatics 1 (IBI1), Applied Data Sciences 2 (ADS2), Genomics and Proteomics 2 (GP2), Infection 3 (IN3), Computational Biology and Systems Biology 3 (CBSB3) and Integrative Biomedical Sciences 4 (IBM4) courses of the Biomedical Informatics and Biomedical Sciences Programmes, Zhejiang University-University of Edinburgh Institute, China, 2020-2023.
  \item Biomedical Disorders 1 and 2 courses of the Integrative Biomedical Sciences PhD programme, Zhejiang University-University of Edinburgh Institute, China, 2021-2023.
  \item Protein bioinformatics part of EMBL Predoc Course, Heidelberg, Germany, 2018.
  \item Basic Teaching Module part of EMBL Predoc Course: Phylogenetics, Tools for 3D molecule interactive visualization and R datasets. Heidelberg, Germany, 2016.
\end{itemize}
\vspace{.1cm}
\textbf{Teaching assistant}
\begin{itemize}
  \item Linear Algebra. National Autonomous University of Mexico. Cuernavaca, Mexico, 2012.
  \item Seminars on Genomic Applications. National Autonomous University of Mexico. Cuernavaca, Mexico, 2011.
  \item Discrete Mathematics. National Autonomous University of Mexico. Cuernavaca, Mexico, 2010 and 2011.
\end{itemize}

\vspace{0.5cm}

\section{Skills and Interests}
\cvitem{Programming languages}{R/Bioconductor, Bash. Basics of: Python, Perl and C.}
\cvitem{Software}{GROMACS, Git, SLURM, Singularity, PyMOL, UCSF Chimera, Modeller, MEGA, IGV, XCMS, Bowtie2, minimap2, samtools, Alfred, \LaTeX}
\cvitem{Languages}{Spanish (native), English (advanced), French (basic), Chinese (entry level)}
\cvitem{Research interests}{Host-Pathogen interactions. Genomics approaches in infectious diseases.}
\cvitem{Other interests}{Traveling, cooking, photography}

%\pagebreak
\vspace{0.5cm}

\section{References}
\begin{itemize}
  \item Dr. Toby, J. Gibson (toby.gibson@embl.de).
  Team Leader at the European Molecular Biology Laboratory, Heidelberg, Germany.
  \item Dr. Bruce M. Russell (b.russell@otago.ac.nz).
  Associate Professor at University of Otago, Otago, New Zealand.
  \item Dr. Jes\'us Mart\'inez-Barnetche (jmbarnet@insp.mx).
  Director and Principal Investigator at the Center for Research on Infectious Diseases (CISEI) of the National Institute of Public Health, Cuernavaca, Mexico.
\end{itemize}

\end{document}
